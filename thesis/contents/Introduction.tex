\section*{Introduction}
\label{Chapter:Intro}
Dynamical systems provide a mathematical framework to analyze, predict and understand the behaviour of systems that co-evolve in time. This formulation encompasses a range of phenomena, including those observed in classical mechanical systems, climate sciences, finance, epidemiology and all other systems that evolve in time. The formulations for such physical systems were traditionally done by making ideal approximations and then deriving simple differential models via Newton's second law of motion. Furthermore, the derivations could often be simplified by exploiting symmetries and different coordinate systems, as highlighted by the success of Lagrangian and Hamiltonian dynamics. The dynamical models obtained through these approaches are known as \textit{first principles models} or \textit{white-box models}.\par
First-principles formulation, however, becomes demanding and sometimes untenable with the increasing complexity of the state interactions, which is often the case with real-world systems of interest. Also, the derived analytical models do not capture any external disturbances. As a result, first principle models often have limited use or poor prediction over longer time spans in the context of control. Nevertheless, a representation of dynamical systems is central to most engineering and scientific applications.\par 
An alternative to first-principles modelling is experimental system identification, i.\,e. obtaining a model from the input and output experimental data of a system. Recently, owing to the increased computational abilities, availability of inexpensive sensors and large storage capabilities, an unprecedented amount of data can be recorded and processed easily. Proven data-driven modelling methods are driving the present-day industry towards data-driven modelling of the complex as well as simple dynamical systems. As Brunton et al. \cite{brunton_kutz_2019} observe:
\begin{displayquote}
Modern dynamical systems is currently undergoing a renaissance, with analytical derivations and first principles models giving way to data-driven approaches. The confluence of big data and machine learning is driving a paradigm shift in the analysis and understanding of dynamical systems in science and engineering. 
\end{displayquote}
% 
% 
\paragraph{Motivation to Obtain Linear Representations of Nonlinear Systems:}
Almost all of the real-world systems of interest exhibit nonlinear behaviour. Obtaining linear representations of such nonlinear systems from data can potentially revolutionize the ability to predict and control these systems as there exist a wealth of techniques for the analysis, prediction, numerical simulation, estimation and control of linear dynamical systems. Thus, whenever possible it is desirable to work with linear dynamics of the form,
% 
\begin{equation}
    \label{eq:Lin}
    \frac{d}{dt}\mathbf{x} = \mathbf{A}\mathbf{x} \;,
\end{equation}
% 
the solution to which is given by 
% 
\begin{equation}
\label{eq:soln}
    \mathbf{x}(t_0+t) = e^{\mathbf{A}t}\mathbf{x}(t_0) \;.
\end{equation}
% 
Such systems admit closed-form solutions and their dynamics are completely characterized by the spectral decomposition of $\mathbf{A}$, also called the \textit{system} matrix:
% 
\begin{equation}
    \mathbf{AT} = \mathbf{T\Lambda}\;,
\end{equation}
% 
where $\mathbf{\Lambda}$ is a diagonal matrix containing the eigenvalues $\lambda_j$ and $\mathbf{T}$ is a matrix whose columns are linearly independent eigenvectors $\xi_j$ associated with eigenvalues $\lambda_j$. In this case, it is possible to write $\mathbf{A} = \mathbf{T\Lambda T^{-1}}$, and the solution in Eq.~\ref{eq:soln} becomes
% 
\begin{equation}
    \mathbf{x}(t_0+t) = \mathbf{T}e^{\mathbf{\Lambda t}}\mathbf{T^{-1}}\mathbf{x}(t_0)\;.
\end{equation}
% 
The matrix $\mathbf{T^{-1}}$ defines a transformation, $\mathbf{z = T^{-1}x}$, into intrinsic eigenvector coordinate, $\mathbf{z}$, where the dynamics become decoupled:
% 
\begin{equation}
    \frac{d}{dt}\mathbf{z} = \mathbf{\Lambda z} \;,
\end{equation}
% 
i.\;e. each coordinate, $z_j$, only depends on itself, with simple dynamics given by
% 
\begin{equation}
    \Dot{z} = \lambda_jz_j \;.
\end{equation}
% 
Thus, it is possible to transform any linear system into eigenvector coordinates where the dynamics become decoupled. No such closed-form solution or simple linear change of coordinates exist in general for nonlinear systems, motivating the search for a linear operator that can express nonlinear dynamics in a linear framework.
% 
% 
\par
Koopman operator theory provides just the framework to represent a nonlinear dynamical system in terms of a linear operator. The Koopman operator advances measurement functions of the state, also called \textit{observables}, to the next time step, thereby making accurate predictions and enabling the possibility of control of those states \cite{spectral_NL}. Recently, Koopman theory has been increasingly applied to system identification \cite{SI_Mezic, mauroy2016linear, proctor2016generalizing, WILLIAMS2016704} estimation \cite{Amit_Surana} and control \cite{MPC_Korda, Abraham} of nonlinear systems. Although the Koopman operator seems like a promising and a simple way to represent nonlinear systems, it must be noted that space spanned by all the measurement functions of the state is usually not bounded and can be described by infinitely many functions of the state. Therefore, the Koopman operator although provides a linear representation of a nonlinear system is infinite-dimensional.\\
 Obtaining a finite-dimensional approximation of the Koopman operator is still an open challenge. Several approaches have been explored to approximate this Koopman operator. Extended Dynamic Mode Decomposition (EDMD) is one such method of approximation where the input-output data is used to regress a finite-dimensional best-fit linear operator that advances measurements of the state or observables forward in time \cite{DMD_book, EDMD_1,Brunton_K_invariant_sub,MPC_Korda}. Sparse Identification of Nonlinear Dynamics (SINDy) \cite{SINDy,brunton_kutz_2019} is another data-driven linear regression approach which leverages the fact that many dynamical systems, described by nonlinear dynamics $\dot{\mathbf{x}} = \textbf{f}(\mathbf{x}),$ have dynamics \textbf{f} with only a few active terms in the space of possible right-hand side functions, and it is possible to solve for only these relevant terms that are active in dynamics using sparse regression. The SINDy algorithm can be easily adapted for classical mechanical systems since there is usually some knowledge of the right-hand side functions of the dynamics. Nevertheless, the success of the above-described methods lies in the choice of observables. This thesis work explores the possibilities of application of data-driven control techniques to an under-actuated nonlinear system, the arm driven inverted pendulum (ADIP). It is of interest to formulate data-driven models and evaluate the performance of data-driven-model-based controllers against that of a first-principles model-based controller.\par

This thesis work is organized as follows: Chapter \ref{Chapter:Prelims} introduces the necessary data regression concepts like Dynamic Mode Decomposition (\ref{sec:DMD}), Koopman operator framework (\ref{sec:KO}), Extended Dynamic Mode Decomposition (\ref{sec:EDMD}), Sparse Identification of Nonlinear Dynamics (\ref{sec:SINDy}) which form the foundation for this work. This chapter also discusses an optimization technique used for tuning the controllers called Genetic Algorithm (\ref{sec:GA}) which is based on natural selection, the process that drives biological evolution. Furthermore, since Section \ref{Sec:firstprinci} deals with the derivation of first-principles model for the ADIP, Chapter \ref{Chapter:Prelims} also covers a brief introduction of the Euler-Lagrange method (\ref{sec:Lagrange}) to obtain governing equations of motion. Chapter \ref{Chapter: Literature} summarises some of the available literature on the application of Koopman operator theory. Chapter \ref{Chapter:Plant} describes the arm-driven inverted pendulum and presents the first-principles model of the same (\ref{Sec:firstprinci}). This chapter also explores the approaches employed to derive a data-driven model of the ADIP (\ref{Chapter:Data}). Chapter \ref{Chapter:Cont} gives a brief introduction of the controllers like the LQR (\ref{sec:lqr}) and MPC (\ref{sec: MPC}), used in this thesis work. Chapter \ref{Chapter:Sim} discusses the simulation of the derived models and compares the performance of the controllers applied to the first-principles model and the data-driven model. Finally, Chapter \ref{Chapter:Conc} summarises the work done in this thesis and discusses the future scope in this area.
\newpage


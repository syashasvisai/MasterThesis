\section{Literature Review}
\label{Chapter: Literature}
In this chapter, a few of the seminal works in the area of application of Koopman operator theory is explored. The following works explore various approaches to obtain the Koopman operator and/or control nonlinear systems through linear Koopman embeddings. Each of these works also demonstrates the application of the derived Koopman operator on various physical systems. 
\par
Abraham et al. \cite{Abraham} explore the application of Koopman operator theory to model and control various robotic systems. In particular, the authors explore the utility of the Koopman operator in deriving a dynamical model of the Sphero SPRK solely from data and evaluate open-loop and closed-loop controller performance of the robot on various terrains. The Sphero SPRK is a differential drive mobile robot enclosed in a spherical case. The underlying mechanism and consequently, the controllers are proprietary and therefore is a good example of applying the data-driven modelling and control strategies. The authors present a generalized way of approximating the dynamical system through linearization of approximated Koopman operator. The linearized Koopman operator is then used for model-based control synthesis. Importantly, this linearized Koopman operator propagates forward only the quantities of interest, for example, the states of the system, thereby limiting the order of the approximated dynamical system while preserving its nonlinear interactions. Furthermore, the authors also explore the possibility of augmenting the Koopman operator to model the interactions of a Vertical Take-Off and Landing (VTOL) drone with an unactuated pendulum attached to it. Only the dynamics of the VTOL are known in this case, and the pendulum is swung up and stabilized based on a good choice of observables and the corresponding data collected. The authors evaluate and compare the overall performance of the system for various choices of observables like polynomial basis and Fourier basis functions. Throughout the work, the authors emphasize the effect of the choice of observables, the number of data points collected and their distribution across the state-space on the fidelity of the generated Koopman Operator.
% 
% 
%
\par
Cisneros et al. \cite{Cisneros.2020} explore Koopman operator-based identification techniques to obtain a velocity based quasi-linear parametric varying (qLPV) model of a control moment gyroscope through the application of EDMD algorithm as illustrated in (\cite{Abraham} and \cite{WILLIAMS2016704}). The qLPV model of the CMG is recovered both in the state-space form (where the knowledge of states is assumed to be available) as well as input/output form (where no knowledge of states is assumed). A model predictive controller is then synthesized for these models. Importantly, the authors exploit a priori knowledge about the plant to construct the vector of observables. The authors also note that even in the absence of a priori knowledge of the system dynamics, when it is possible, observables derived from the intuitive knowledge of the system and physical insight perform much better than generalized basis functions like radial basis functions (RBFs) used by Korda et al. Importantly, the approximate Koopman operator here was computed online. This was done by running a short open-loop experiment using chirp test signals to boot-strap the Koopman algorithm and give a meaningful initial model to the MPC controller. The Koopman operator was updated online recursively subject to certain threshold constraints where the Koopman operator was updated only when novel dynamics recognized by significant prediction errors were encountered. Furthermore, the authors compared the closed-loop performances of the model recovered through velocity based qLPV and the one recovered through direct Koopman approach by Korda et al. \cite{MPC_Korda}, on the same tracking scenario. The authors noted that the former model which has fewer states (and therefore easier to tune) consequently performed much better in tracking a trajectory than the latter which had many more lifted states and also was highly sensitive to any perturbations in the tuning weights. The higher number of lifted states also meant that the computational complexity of the model recovered through direct Koopman approach was significantly higher than the other. The authors thus conclude that converting high dimensional Koopman model via velocity linearization into a much lower dimensional qLPV system and then using the qLMPC approach proposed by Cisneros et al. in \cite{Cisneros2016}, led to superior control performance and numerical efficiency over the methods used by Korda et al. The above claims were corroborated through experimentation on a control moment gyroscope.
% 
% 
% 
\par
Korda et al. \cite{MPC_Korda} extend the definition of Koopman operator to controlled dynamic systems by viewing the controlled dynamical system as an uncontrolled one evolving on an extended state-space given by the product of the original state-space and the space of all control sequences. A modified version of the extended dynamic mode decomposition is then used to compute a finite-dimensional approximation of the Koopman operator with control. The authors test the linear predictors (linear A, B and C matrices) obtained through this approach on a Van der Pol oscillator to predict the evolution of the system dynamics and conclude that these linear predictors exhibited a superior predictive performance as compared to the one obtained by Carleman linearization and local linearization methods. Additionally, the authors also extend the application of these linear predictors to synthesize a model predictive feedback controller. Importantly, the authors show that the computational complexity of the model predictive controllers synthesized based on these linear predictors was comparable to that of linear dynamical systems with the same number of control input and states. This was achieved by using \textit{dense-form} of an MPC problem whose computational complexity is independent of the number of states (or the dimension of the lifted system) and only dependent on the number of control inputs. This was experimentally demonstrated by comparing the performance of feedback control of a bilinear motor in tracking a reference trajectory. Moreover, the comparison is made between an MPC synthesized for a model derived solely from data with no knowledge of the states and an MPC synthesized for a model where the exact dynamics were known and linearized locally.
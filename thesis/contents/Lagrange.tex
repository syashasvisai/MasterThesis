\subsection{Euler-Lagrange Formulation}
\label{sec:Lagrange}
The \textit{equations of motion} describe how a rigid body or a system moves under the influence of a force as a function of time. There are several ways to derive the equations of motion, the most common being applying Newton's second law of motion. While this method's application is simple and straight-forward for systems with Cartesian coordinate systems and/or low dimensional systems, the complexity of the computations increases with the dimension of the system or if the coordinate system is not perpendicular. The go-to solution for formulating the equations of motion in such cases is the \textit{Euler-Lagrange} formulation based on energy method. Unlike the Newtonian approach, which involves vector quantities, the Lagrangian approach involves scalar quantities such as the kinetic energy and the potential energy and proves to be easier than the Newtonian approach for most systems.\par
Lagrangian formulation involves describing a rigid body system in terms of \textit{generalized coordinates}. These generalized coordinates can be chosen freely as per the requirement. They can be position coordinates, angles, momentum of particles, or charges. There can even be additional dependent generalized coordinates to help in formulating effective control laws. \par
A brief procedure of deriving the governing equation of motion through Lagrangian formulation based on the energy method can be given as follows:
\begin{enumerate}

    \item Choose a set of generalized coordinates $\mathbf{q} \in \mathbb{R}^n$ that completely describes the system configuration.
    
    \item Define a set of generalized forces $\mathbf{u} \in \mathbb{R}^m$.
    
    \item Compute the kinetic energy $T(q,\Dot{q})$ and the potential energy, $V(q)$ of the system.
    
    \item Compute the Lagrangian function $ L(q,\Dot{q})$ defined as the difference between the overall kinetic energy of the system and the potential energy of the system.
        \begin{equation}
        L = T - V\;,
        \end{equation}
        
    \item The equations of motion can now be expressed in terms of the Lagrangian as follows:
    \begin{equation}
    \label{eq:LagrangeEqn}
    \frac{d}{dt}\left(\frac{\partial{L}}{\partial{\Dot{q}}}\right) - \frac{\partial{L}}{\partial{q}} = u\;.     
    \end{equation}
\end{enumerate}
\par

The dynamical model formulated using the Lagrange formulation method can then be written in the following form:
\begin{equation}
\label{eq:lagdyn}
    \mathbf{M}\mathbf{\Ddot{q}} + \mathbf{C(q,\Dot{q})}\mathbf{\Dot{q}} + \mathbf{G(q)} = \mathbf{u}\;,
\end{equation}
where $\mathbf{\Ddot{q}} = [\Ddot{q}_1, \Ddot{q}_2, \dots, \Ddot{q}_n]^{\top}$ is the vector of accelerations, $\mathbf{\Dot{q}} = [\Dot{q}_1, \Dot{q}_2, \dots, \Dot{q}_n]^{\top}$ is the vector of velocities, and $\mathbf{q} = [q_1, q_2, \dots, q_n]^{\top}$ is the vector of generalized coordinates.  $\mathbf{M(q)}$ is the inertia matrix, $\mathbf{C(q,\Dot{q})}$ is the Coriolis and centripetal force matrix, and $\mathbf{G(q)}$ is the gravitational force matrix. $\mathbf{u} = [u_1, u_2, \dots, u_m]^{\top}$ is the vector of generalized forces.
\pagebreak
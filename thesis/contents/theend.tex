\section{Conclusion and Future Scope}
\label{Chapter:Conc}
This thesis work focuses on the merits and demerits of implementing the Koopman operator theory for model-based control. The central idea of the Koopman operator framework is to lift the nonlinear dynamics of a system to a higher dimensional space where the evolution of dynamics becomes approximately linear. Such linear data-driven models were derived using known regression techniques like DMD~(Sec.~\ref{sec:DMD}) and EDMD~(Sec.~\ref{sec:EDMD}). Model-based controllers were then synthesized for such data-driven models, and the performance of the controllers was evaluated on a trajectory tracking problem. The data-driven models on a whole exhibit superior performance as compared to locally linearized models and moreover, linear control design methods could be readily applied to these models. It was observed that increasing the `complexity' of candidate functions, or in simple words, augmenting the state vector with nonlinear functions of the state further improved the performance of data-driven approximation models. For example, in the open-loop identification~(Sec~\ref{sec:Results_ident}), with the increasing complexity of the candidate functions, the SINDy regression algorithm could predict the state's evolution over a longer time period and for different operating conditions. It was also found that the prediction could be further improved when the data are sampled from various trajectories. The flexibility of data approximation methods allows for data from multiple trajectories and missing data points to be used. Furthermore, it was observed that the volume of data and its distribution across the state space has a significant effect on the computation of the Koopman operator.  \par
These vital observations obtained from open-loop identification were applied in recording the data for closed-loop identification since the Up-Up configuration is where the focus of this thesis work lies. The closed-loop performance of the data-driven controllers was compared to that of the locally linearized controllers. To this end, a first-principles model was derived~(Sec.~\ref{Sec:firstprinci}) and locally linearized by Jacobian linearization at the Up-Up position. Both a non-predictive controller such as the linear quadratic regulator and a predictive controller such as the model predictive controller were synthesized to compare the tracking performance in combination with the data-driven model. It was observed that non-predictive controllers do not perform as well as predictive controllers when the state is augmented with certain nonlinear functions of the state. It is hard to predict which functions might fail non-predictive controllers, and therefore, when working with data-driven models, it is recommended to work with predictive controllers. Moreover, the tracking performance of the predictive controller was far greater than the non-predictive controller~(Tab.~\ref{tab:RMSE_2}). And, with the recent advancement in computational capabilities, predictive control is much less cost-intensive than previously. Also, it must be noted that the computational complexity of a high-dimensional lifted system is rendered virtually independent of the dimension of the state by formulating the MPC problem using the so-called dense-form \cite{MPC_Korda}. Additionally, with a judicious choice of observables, it was observed that the data-driven models could capture nonlinearities that are otherwise not captured by locally linearized models. As a result, the tracking range of data-driven model-controllers could be significantly improved over the locally linearized model-controllers or even other data-driven models that are approximated with only measurements of state~(Fig.~\ref{fig: KPMC_comp}).\newpage
The robustness of tracking performance was tested for the data-driven model approximated with nonlinear functions of the state~(Fig.~\ref{fig: cont_traj}). It was observed that as the slope of trajectory became steeper, the performance of the data-driven model controller decreased. There could be two potential reasons for this. First, the set of observables or the nonlinear functions of the state may not be optimal in the sense of capturing the necessary nonlinearities, and there may exist some other combination of observables that perform better. The possibility of finding these `best' observables is left to the scope of future research. Second, the data-driven model's performance degradation may also be construed as a possible limitation of data-driven models. It must be noted that `extended' tracking range here means that the controller can track a trajectory beyond the range of trajectories used for training the data-driven model. Although the controller could successfully track an extended trajectory in a fairly good manner, as the slope of the trajectory increased, the controller could not perform as efficiently as desired. This could mean that the state-space in which the data-driven controllers can perform equally well or even better than the locally linearized first-principles model-controllers is limited to the range of training data. Nevertheless, the possibility of extending the range of a controller by simply augmenting the state with a wise choice of observables seems to be a promising area of research.\par
Moreover, the data-driven model was compared with a quasi-linear parameter varying model~(Fig.~\ref{fig: cont_traj_qLMPC}). Some interesting insights could be derived from this comparison. Firstly, although the qLMPC is a computationally expensive process compared with Koopman MPC, the computation time was well within the acceptable hardware limit. Second, the relative RMSE for a qLMPC was significantly lesser than that of the Koopman MPC. Moreover, Cisneros et al. \cite{qLMPC} were able to achieve up to $80^\circ$ as the maximum range of tracking. Therefore, it can be concluded that the first-principles model-based qLMPC performs much better than the Koopman MPC, which heavily depends on the choice of observables. However, one major assumption that is made here is the availability of the first-principles model. In the absence of a first-principles model, the Koopman model is a powerful and useful approximation of the underlying nonlinear dynamics. Thus, it can be said that the choice of model depends on the particular use case. \par
Finally, two different strategies to swing-up the pendulum from its stable equilibrium and balance it about its unstable equilibrium were evaluated. This comparison was made to investigate the possibility of swing-up using MPC controller and evaluate it against a proven swing-up law. The energy-based swing-up law proposed by Fantoni et al.~\cite{Fantoni} could achieve successful swing-up in just one swing as the swing-up law ensures the convergence of the pendulum trajectory to a homoclinic orbit around the unstable equilibrium. When the pendulum enters this homoclinic orbit, the controller is switched from an energy-based controller to a stabilizing controller such as the LQR to stabilize the pendulum about its unstable equilibrium. This is unlike the second strategy which employs the qLPV approach to linearize the plant at every time step and synthesize a corresponding model predictive controller. Although both the controllers achieve swing-up and stabilization in approximately the same time~(Sec.~\ref{sec: swingup}), it must be noted that the former approach employed two controllers, one for swing-up and one for stabilizing the pendulum, whereas the latter employs only a single controller to achieve both the swing-up and stabilization tasks.
\newpage
\subsubsection*{Future Scope}
In this thesis work, the observables, especially the RBFs, were chosen at random from a defined set. Although it became evident that a fortuitous set of observables significantly improved the performance of a tracking controller, the case for data-driven controllers can be strengthened if one could find a mathematical approach to exactly identify such observables. The SINDy algorithm is a good start in this research direction. One could also utilise any prior information of the particular dynamical system to discover such observables. However, one must think of a way to validate any such potential observables obtained based on physical insight or intuitive knowledge of the dynamics. Future work should also focus on proving or imposing closed-loop stability guarantees. An interesting direction for future work of this particular thesis work can be in the area of online-learning or active-learning, wherein a new Koopman model is learned online whenever novel dynamics are encountered. This is similar to a linear parameter varying model based on first-principles and has been explored for a Control Moment Gyroscope by Cisneros et al.~\cite{Cisneros.2020}.
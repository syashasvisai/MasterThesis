\subsection{Extended Dynamic Mode Decomposition}
\label{sec:EDMD}
The extended dynamic mode decomposition(EDMD) first introduced by Williams et al. \cite{WILLIAMS2016704} aims to approximate the Koopman operator and therefore the Koopman eigenvalue, eigenfunction and modes solely from data and can be considered as an extension to the DMD which approximates only the Koopman eigenvalue and modes. The EDMD also allows for nonlinear measurements of state to be included in the vector of observables to better approximate the dynamical interactions. Thus, the system is first \textit{lifted} into a higher dimensional space that is defined by a set of nonlinear functions. The original state space can also be included in this higher dimensional space, thereby making the best of both worlds i.\,e. obtaining a nearly accurate linear representation of the nonlinear dynamics and the availability of the original state space for control, as will be illustrated in the next chapters. An important extension of the EDMD is the EDMD with control (EDMDc) which enables incorporation of input measurements into the vector of observables to disambiguate the effect of internal dynamics from actuation or external inputs \cite{DMDC}. This method is beneficial in identifying actuated dynamical systems and enable closed-loop feedback control, for example, it has been used by Korda et al. for feedback control of a bilinear motor through Koopman driven model predictive control \cite{MPC_Korda}.
\par
The EDMD approach requires only
\begin{enumerate}
    \item a data set of snapshot pairs of the system state, $\{\mathbf{x}_k,\mathbf{x}_{k+1}\}_{k=1}^M$ arranged as in Eqs. \ref{data1} and \ref{data2}, and
     \item a \textit{dictionary} of observables, $\mathcal{D} = \{\psi_1, \quad \psi_2, ~ \dots,~ \psi_K\}$ where $\psi_i \in \mathcal{F}$, whose span is denoted as $\mathcal{F_D} \subset \mathcal{F}$. $\mathcal{D}$ must be ``rich enough'' to accurately approximate a few of the leading Koopman eigenfunctions.
\end{enumerate}
Although the selection of observables which most closely approximate the dynamics of a system is still an open question, there are several approaches to choose the best possible observables. For example,
\begin{itemize}
    \item they can be selected based on physical insights of the system,
    \item partial or complete knowledge of the right hand side functions of the dynamical system,
    \item as Fourier basis functions, Radial basis functions, Hermite polynomials etc.
\end{itemize}
The list is not exhaustive. However, it is always advised to run a quick check to validate the behaviour of the system for the choice of observables. It is possible that the evolution of some observables is not bounded, resulting in an infinite-dimensional linear operator. 
% An example of this can be found in the appendix.
\newpage
\textbf{Approximating the Koopman operator through the EDMD approach \cite{WILLIAMS2016704}:} 
\begin{itemize}
    \item The state measurements are collected as detailed in Eqs. \ref{data1} and \ref{data2}.
    \item A vector of observables defined by functions of state measurements is constructed to represent the dynamics of the system in terms of a  \textit{lifted} state, $\mathbf{\Psi(x)}:\mathcal{M} \rightarrow \mathbb{C}^{1 \times K}$,
    % 
    \begin{equation}
    \label{Eq: DataMatrix}
        \mathbf{\Psi(x)} = [\psi_1(\mathbf{x}) \quad \psi_2(\mathbf{x}) ~ \dots ~ \psi_{K}(\mathbf{x})] \;.
    \end{equation}
    % 
    The data set of snapshots is typically constructed from multiple short bursts of simulation or experimental data. $\mathbf{\Psi(\mathbf{x})}$ may contain the original state $\mathbf{x}$ as well as nonlinear measurements, so often $K\gg n$. For this thesis work, the scalar measurements of the state form the first $n$ entries of $\mathbf{\Psi(x)}$ and the remaining $(K-n)$ observables are nonlinear functions of the state measurements. 
    \item The EDMD then seeks to approximate a linear operator $\mathbf{\mathcal{A}} \in \mathbb{R}^{K\times K}$ that approximately relates the lifted measurements of state, at least for short periods of time:
    % 
    \begin{equation}
        \mathbf{\Psi(x}_{k+1}) \approx \mathbf{\mathcal{A}\Psi(x}) \:.
    \end{equation}
    % 
    By definition, a function $\textsl{g} \in \mathcal{F_D}$ can be written as
    \begin{equation}
        \textsl{g} = \sum_{j=1}^K a_j\psi_j = \mathbf{\Psi a} \;,
    \end{equation}
    the linear representation of the $K$ elements in the dictionary with the weights $\mathbf{a}.$ Because $\mathcal{F_D}$ is typically not an invariant subspace of $\mathcal{K}$,
    \begin{equation}
        \mathcal{K}\textsl{g} = (\mathbf{\Psi} \circ \mathbf{F})\mathbf{a} = \mathbf{\Psi(\mathcal{A}a)} + r
    \end{equation}
    which includes the residual term $r\in \mathcal{F}$. $\mathbf{\mathcal{A}}$ is determined by minimizing, 
    % \item The relation described by Eq.~\ref{eq:disc_KO} can then be given as
    % \begin{equation}
    %     \mathbf{\Psi(x}_{k+1}) = \mathbf{\Psi(x}_{k})\mathbf{\mathcal{A}} + r(\mathbf{x}_k) \;, 
    % \end{equation}
    % where $ \mathbf{\mathcal{A}} \in \mathbb{C}^{N \times N}$ and $r(\mathbf{x}_k)$ is the approximation error or the residual.
    % \item The matrix $\mathbf{\mathcal{A}}$ which is the EDMD operator or approximation of the Koopman operator restricted to space spanned by the observables can be computed in many ways. Here, we use the EDMD approach described by Williams et al. \cite{WILLIAMS2016704}, where $\mathbf{\mathcal{A}}$ is determined by minimizing
    \begin{align}
    \label{eq:KO_residual}
    \begin{split}
        J &= \frac{1}{2}\sum_{k=1}^{M} ~\lvert r(\mathbf{x}_k) \rvert ^2 \;,\\
          &= \frac{1}{2}\sum_{k=1}^{M} ~\lvert ((\mathbf{\Psi \circ F})(\mathbf{x}_k) - \mathbf{\Psi(x}_{k})\mathbf{\mathcal{A}})\mathbf{a} \rvert ^2\\
          &= \frac{1}{2}\sum_{k=1}^{M} ~\lvert (\mathbf{\Psi(x}_{k+1}) - \mathbf{\Psi(x}_{k})\mathbf{\mathcal{A}})\mathbf{a} \rvert ^2 \;.
    \end{split}
    \end{align}
    Eq. \ref{eq:KO_residual} is a least squares problem, and therefore cannot have multiple isolated local minima; it must either have a unique global minimizer or a continuous family of minimizers. As a result, regularization may be required to ensure the solution is unique and the $\mathbf{\mathcal{A}}$ that minimizes (\ref{eq:KO_residual}) is:
    \begin{equation}
    \label{eq:EDMD1}
        \mathbf{\mathcal{A}} = \mathbf{G}_1^{\dagger}\mathbf{G}_2\;,
    \end{equation}
    where
    \begin{align}
    \label{eq:EDMD2}
        \mathbf{G}_1 & = \frac{1}{M}\sum_{k=1}^{M} ~\mathbf{\Psi(x}_{k})^*\mathbf{\Psi(x}_{k}) \;, \\
    \label{eq:EDMD3}
        \mathbf{G}_2 & =\frac{1}{M}\sum_{k=1}^{M} ~\mathbf{\Psi(x}_{k})^*\mathbf{\Psi(x}_{k+1})\;.
    \end{align}
    with $\mathbf{\mathcal{A}},\mathbf{G}_1,\mathbf{G}_2 \in \mathbb{C}^{K\times K}$. Thus, $\mathbf{\mathcal{A}}$ is a finite dimensional linear operator that maps $\textsl{g} \in \mathcal{F_D}$ to some other $\hat{\textsl{g}} \in \mathcal{F_D}$ by minimizing the residuals at the data points. It is interesting to note that $\mathbf{\mathcal{A}}$ is the transpose of the finite dimensional approximation of the Koopman operator and is exactly equal to the finite dimensional Koopman operator approximated through DMD on special set of measurements: the measurements of full-state. As a consequence, if $\boldsymbol{\xi}_j$ is the $j$-th right eigenvector of $\mathbf{\mathcal{A}}$ with the eigenvalue $\lambda_j$, then the EDMD approximation of an eigenfunction of $\mathcal{K}$ is
    \begin{equation}
        \varphi_j = \mathbf{\Psi}\boldsymbol{\xi}_j\;.
    \end{equation}
    And consequently a vector valued function $\mathbf{\Phi} : \mathcal{M}\rightarrow \mathbb{C}^K$ where,
    \begin{equation}
        \mathbf{\Phi(x)} = [\varphi_1(\mathbf{x}), \varphi_2(\mathbf{x}), \dots, \varphi_K(\mathbf{x})]
    \end{equation}
    can be written as:
    \begin{equation}
        \mathbf{\Phi(x)} = \mathbf{\Psi(x)}\mathbf{\Xi}, \quad \quad \mathbf{\Xi} = [\boldsymbol{\xi}_1, \boldsymbol{\xi}_2,\dots,\boldsymbol{\xi}_K] \;.
    \end{equation}
    The computational burden of this approach grows as the dimension of $\mathbf{\Psi(x)}$ increases. However, also, this approach generally yields a better approximation as the dimension of $\mathbf{\Psi(x)}$ increases. Therefore, the choice of observables is crucial. Furthermore, the number of data points and their distribution across the state-space will have a large effect on the computed $\mathbf{\mathcal{A}}$ matrix, since the optimality of the approximation holds only in the region of the obtained data.
    \item EDMD also computes the approximation of Koopman modes for the full state observable. Here, only the final result is presented. An in detail derivation can be found in \cite{WILLIAMS2016704}.  The full state observable can be expressed as,
    % 
    \begin{equation}
        \textsl{\textbf{g}}(\mathbf{x}) = \boldsymbol{V}\mathbf{\Psi}(\mathbf{x})^\top = \sum_{j=1}^K~\boldsymbol{v}_j\varphi_j(\mathbf{x}), \quad \quad \boldsymbol{V} = [\boldsymbol{v}_1, \boldsymbol{v}_2, \dots, \boldsymbol{v}_K] = \mathbf{W^*B}^\top
    \end{equation}
    % 
    where $\boldsymbol{v}_i = (\mathbf{w}_i^*\mathbf{B})^\top$ is the $i$-th Koopman mode and $\mathbf{w}_i$ is the $i$-th left eigenvector of $\mathbf{\mathcal{A}}$ associated with $\lambda_i$, and $\mathbf{B} = [\mathbf{b}_1,\mathbf{b}_2,\dots,\mathbf{b}_n], \mathbf{B} \in \mathbb{C}^{K \times n}$ is some appropriate vector of weights:
    \begin{equation*}
        \textsl{g}_i(\mathbf{x}) = \sum_{j=1}^K~\psi_j(x) b_{j,i} = \mathbf{\Psi(x)b}_i.
    \end{equation*} 
\end{itemize}
 \newpage
 In summary, EDMD computes a finite dimensional approximation of the Koopman operator from snapshot pairs of a data set, a dictionary of observables and the assumption that the leading Koopman eigenfunctions are (nearly) contained within $\mathcal{F_D}$. Furthermore, the eigenvalues of $\mathbf{\mathcal{A}}$ are the EDMD approximations of the Koopman eigenvalues, the right eigenvectors of $\mathbf{\mathcal{A}}$ generate the approximations of the eigenfunctions and the left eigenvectors of $\mathbf{\mathcal{A}}$ generate the approximations of the Koopman modes.
 \par
As previously discussed, the EDMD approach can readily be extended to actuated systems of the form
% 
\begin{equation}
\label{eq:EDMDc}
    \mathbf{x}_{k+1} = \mathbf{A}\mathbf{x}_k + \mathbf{Bu}\;,
\end{equation}
to estimate the approximate $\mathbf{A}$, and $\mathbf{B}$ matrices. The approach employed is largely similar to the one described above. The state measurements are first lifted to a higher dimensional $\mathbf{\Psi(x)}$ and additionally, the corresponding input measurements are recorded as
% 
\begin{equation}
    \label{Eq: InputMatrix}
    \mathbf{u} =  \begin{bmatrix}
        |                &  |               &    \quad&         |         \\
        \mathbf{u}_{1}   &  \mathbf{u}_{2}, &   \dots & \mathbf{u}_{m} \\
        |                &  |               &    \quad&         |       
        \end{bmatrix}  \;.
\end{equation}
% 
$\mathbf{\Gamma(u}_k)$ can simply be scalar measurements of input.  For the purpose of this thesis work, $\mathbf{\Gamma} \in \mathbb{R}^{m}$, is a vector of scalar measurements of input.
The state observables vector, (\ref{Eq: DataMatrix}), and the input vector, (\ref{Eq: InputMatrix}), are then combined to form an augmented observables vector, $\mathbf{\Omega} \in \mathbb{C}^{(K+1) \times {m}} $
% 
\begin{equation}
    \mathbf{\Omega(x,u)} = [ \mathbf{\Psi} \quad \mathbf{\Gamma}]^\top \;.
\end{equation}
%
The EDMDc operator then seeks to approximate the system in Eq. \ref{eq:EDMDc}:
% 
\begin{equation}
    \mathbf{\Omega}(\mathbf{x}_{k+1},\mathbf{u}_{k+1}) = [\mathbf{\hat{A}~~\hat{B}}]\begin{bmatrix}
    \mathbf{\Psi}_k \\ \mathbf{\Gamma}_k
    \end{bmatrix}\;,
\end{equation}
% 
where the approximate system matrix $\mathbf{\hat{A}} \in \mathbb{R}^{K\times K}$ and the approximate input matrix $\mathbf{\hat{B}} \in \mathbb{R}^m$ can be easily computed by decomposition of the Koopman operator in (\ref{eq:EDMD1}) obtained by replacing $\mathbf{\Psi(x_k)}$ with the corresponding augmented observables vector $\mathbf{\Omega(x_k,u_k)}$ in Eqs.~[\ref{eq:EDMD1}~-~\ref{eq:EDMD3}].
\par
In summary, the enriched dictionary of observables $\mathcal{D}$ provides a larger basis in which the Koopman operator can be approximated. It has been shown that in the limit of infinite snapshots, the EDMD operator converges to the Koopman operator projected onto the subspace spanned by $\mathcal{D}$ \cite{Korda_edmd_conv}. However, it must be noted that if $\mathcal{D}$ does not span a Koopman invariant subspace, then the projected operator may not have any resemblance to the original Koopman operator, as all of the eigenvalues and eigenvectors may be different. In fact, it was shown that EDMD operator would have spurious eigenvalues and eigenvectors unless it is represented in terms of a Koopman invariant subspace \cite{Brunton_K_invariant_sub}. Therefore, it is essential to use validation and cross-validation techniques to ensure that the EDMD models are not overfitting. This is possible by leveraging the SINDy regression \cite{SINDy} to identify Koopman eigenfunctions corresponding to a particular eigenvalue $\lambda$, selecting only a few active terms in the dictionary $\mathcal{D}$ to avoid overfitting. 
\newpage
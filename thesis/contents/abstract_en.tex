This work investigates the Koopman operator framework's implementation to approximate linear predictors for the Arm-Driven Inverted Pendulum, a nonlinear controlled dynamical system. The Koopman approach promises to apply the tools developed for linear dynamical systems to the dynamical systems defined by the Koopman operator; thus obtaining a linear approximation of a nonlinear system without directly linearizing around a fixed point. This is done by lifting or embedding the underlying nonlinear dynamics into a higher dimensional space where the Koopman operator's evolution is approximately linear. Although linear, the Koopman operator is infinite-dimensional, which renders it unsuitable for application purposes and therefore, a finite-dimensional approximation is computed by applying the Extended Dynamic Mode Decomposition. Such models' tracking performance is evaluated against a model obtained from first-principles linearized around a fixed point. It is seen that the Koopman model exhibits performance superior to a locally linearized model. Importantly, the approach to compute such a model is completely data-driven and extremely simple. It involves only a nonlinear transformation of the measured data (also called \textit{lifting} the dimension of the data) and a linear least-squares problem in the higher dimensional lifted space that can be readily solved even for large data sets. This approximated linear Koopman model can be readily used to synthesize controllers for the underlying nonlinear dynamical system using linear control design methodologies. This work also compares the performance of non-predictive controllers obtained in this way against that of predictive controllers. Predictive controllers such as the Model Predictive Controller offers numerous advantages over non-predictive controllers such as the Linear Quadratic Regulator, more so in the case of data-driven control.
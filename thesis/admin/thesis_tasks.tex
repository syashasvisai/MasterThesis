 % Project description
\def\projectdescription{
The Koopman operator theory \cite{Koopman1931}, is an increasingly popular formalism of dynamical systems theory that enables analysis, prediction and control of nonlinear dynamics from measurement data. Such data-driven techniques allow for devising linear control strategies for highly nonlinear dynamical systems for which first principle models are not available or are impractical to be constructed.\par
The motivation for this Master's thesis is derived from the desire to generate dynamical models of an ADIP through measured data. In particular it is of interest to synthesize linear model-based controllers using these data-driven models in simulation and experimentally compare the performance of data-driven controllers with LPV contrrollers based on first principles in tracking a reference trajectory.
}

 % Project tasks
\def\tasks{
\begin{packedenumerate}
	\item Literature review of state-of-the-art architecture for data-driven control systems
	\item Develop a data-driven approach using Koopman operator theory for constructing a linear representation of the ADIP plant for the purpose of swing-up and stabilizing control
    	\item Synthesize a quasi-LPV controller governing the swing-up dynamics and a stabilizing Linear Quadratic Regulator and a stabilizing quasi-LPV controller for the aforementioned data-driven model
    	\item Simulate and compare the performance of the derived controllers in tracking a reference trajectory
    	\item Experimentally validate the simulation results
    	\item Optional: Extend the above framework to synthesizing a stabilizing data-driven quasi-LPV Predictive Controller using Koopman operator techniques(KqLMPC) and compare it with the existing results on the performance of qLMPC control strategy applied on the model obtained from first principles, \cite{qLMPC}. 

\end{packedenumerate}
}